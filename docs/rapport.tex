\documentclass[a4paper]{article}

\title{Projet PLE \\ Univerité de Bordeaux 2020-2021}
\author{
    Alexandre Erard, \\
    Maxime Gresse
    }    

\begin{document}
\maketitle
\section{Infrastructure}
\begin{itemize}
    \item Les utilisateurs de twitter produisent environs $504.10^6$ tweets par jours. Sachant que 1 tweet fait environ 5Ko.
    \[ 504.10^6 \times 5 \times 3 = 756.10^7 Ko = 7,56 To\]
    Notre Cluster possède actuellement 18.16To de stockage. Donc on peut stocker:
    \[  \frac{18,16}{7,56} \approx 2,40\]
    un peu plus de 2 jours de tweet.
    
    \item Pour 2 jour de tweet on obtient un total de $7,56 \times 2 = 15To$. Un blocs faisant 128Mo on obtient 117187 blocs disponnibles.
    \[ \frac{1,5.10^7}{128} \approx 117187\]

    \item 5ans de tweets équivaut à  $7,56 \times 365 \times 5 = 13797To $ de données.
    Nous disposons actuellement de 20 machines possédant en moyenne 
    \[\frac{18.16}{20} = 0.93 = 930Go\]
    En gardant un stockage de 930Go par machine il nous faudrait alors 14816 machines.
    \[ \frac{13797-18.6}{0.93} \approx 14815.5\]
    

\end{itemize}

\end{document}